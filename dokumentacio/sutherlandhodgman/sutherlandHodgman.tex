\documentclass[12pt]{article}
\usepackage{mathtools,amssymb, amsthm, tikz}
\usetikzlibrary{intersections}
\usepackage[margin=1in]{geometry}
\usepackage{url}
\usepackage{natbib}
\usepackage[colorlinks=true, linkcolor=black, urlcolor=blue, citecolor=blue]{hyperref}
\usepackage[T1]{fontenc}
\usepackage[utf8]{inputenc}
\usepackage{lmodern}
\usepackage{fontspec}
\setmainfont{Times New Roman}

\title{Sutherland-Hodgman algoritmus}
\author{Pintér Bálint}
\renewcommand{\contentsname}{Tartalomjegyzék}
\renewcommand{\bibsection}{\section*{Források}}


\begin{document}
\maketitle
\newpage
\tableofcontents
\newpage

\section{Bevezetés}
A Sutherland-Hodgman algoritmus egy clipping algoritmus. Feladata, hogy levágja
az alakzatok látómezőről lelógó részeit. A teljesen kívül eső alakzatokat
elveti, a kilógókat úgy vágja meg, hogy csak a látható rész maradjon.
\begin{center}
    \begin{tikzpicture}
        \coordinate (A) at (0,0);
        \coordinate (B) at (5,0);
        \coordinate (C) at (5,3);
        \coordinate (D) at (0,3);
        \coordinate (E) at (-2,-1.5);
        \coordinate (F) at (2,1);
        \coordinate (G) at (-1,2);

        \draw[name path=rectangle] (A) -- (B) -- (C) -- (D) -- cycle;
        \draw[name path=triangle] (E) -- (F) -- (G) -- cycle;
        \path [name intersections={of=rectangle and triangle, by={H, J}, sort by=rectangle}];
        \fill [black] (H) circle (3pt) node[below right] {$H$};
        \fill [black] (J) circle (3pt) node[below left] {$J$};

        \node[left] at (A) {$A$};
        \node[below right] at (B) {$B$};
        \node[above right] at (C) {$C$};
        \node[above left] at (D) {$D$};

        \node[left] at (E) {$E$};
        \node[right] at (F) {$F$};
        \node[above] at (G) {$G$};

    \end{tikzpicture}
    \\
    \textbf{1. Ábra:}\\Sutherland-Hodgman algoritmus szemléltetése
\end{center}
\newpage
\nocite{*}
\bibliographystyle{plainnat}
\bibliography{forrasok}

\end{document}