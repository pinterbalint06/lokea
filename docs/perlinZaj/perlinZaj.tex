\documentclass[12pt]{article}
\usepackage{mathtools}
\usepackage[margin=1in]{geometry}
\usepackage[T1]{fontenc}
\usepackage[utf8]{inputenc}
\usepackage{lmodern}
\usepackage{fontspec}
\setmainfont{Times New Roman}

\usepackage[linesnumbered,ruled,plain,longend]{algorithm2e}

\newcommand{\assign}{\leftarrow}

% Algoritmus név
\SetAlgorithmName{Algoritmus}{Algoritmusok}{Algoritmus}
\SetAlgoCaptionSeparator{:} 
\makeatletter
\renewcommand{\fnum@algocf}{\AlCapSty{\AlCapFnt\thealgocf.\nobreakspace\algorithmcfname}}
\makeatother

% Függvények
\SetKwProg{Eljaras}{Eljárás}{:}{Eljárás vége}
\SetKwProg{Fuggveny}{Függvény}{:}{Függvény vége}

% Változók
\SetKwInput{Valtozok}{Változók}
\SetKwInput{Konstansok}{Konstansok}
\SetKwInput{Be}{Be}
\SetKwInput{Ki}{Ki}

% Ciklusok
\SetKwFor{Ciklus}{Ciklus}{}{Ciklus vége}
\SetKwFor{CiklusAmig}{Ciklus amíg}{}{vége}

% Elágazás
\SetKwIF{Ha}{}{Kulonben}{Elágazás}{akkor}{Különben}{}{Elágazás vége}

% Logikai operátorok
\SetKw{Es}{$\wedge$}
\SetKw{Vagy}{$\vee$}
\SetKw{Nem}{$\neg$}

% Tömb
\SetKw{Tomb}{tömb}

\title{Perlin-zaj}
\author{Pintér Bálint}
\renewcommand{\contentsname}{Tartalomjegyzék}

\begin{document}
\maketitle
\newpage
\tableofcontents
\newpage

\section{Perlin-zaj}
A Perlin-zaj egy olyan zaj, amely sima természetesen változó zajt generál. Így
használható természetbeli dolgok leszimulálására mint például egy domborzat.
Bármennyi dimenzióra létrehozható, de jellemzően az elsőtől a negyedik
dimenzióig alkalmazzák. A mi kódunkban egy két dimenziós Perlin-zaj van
implementálva.
\section{Előkészítés}
A Perlin-zajnak szüksége van egy gradiens táblára és egy permutációs táblára.
\subsection{Gradiens tábla}
A gradiens tábla vektorokat tárol a zaj dimenzióinak megfelelően. (Két
dimenziós zaj -> két dimenziós vektor)
\subsection{Permutációs tábla}
A permutációs tábla kezdetileg 0-tól 255-ig tartalmazza a számokat, majd ezeket
összekeverjük és megduplázzuk (így egy 512 elemű tömböt kapunk). Így a
hashelésnél nem kell odafigyelni a túlindexelésre.
\begin{algorithm}
    \caption{Permutációs tábla létrehozása}
    \DontPrintSemicolon

    \Eljaras{PermutaciosTablaGeneral (PermutaciosTabla, MaxP, Rand)}{

        % Változók
        \Konstansok{MaxP=512}

        \Ciklus{$i \coloneqq 1$-től $256$-ig}{
            PermutaciosTabla[$i$] $\coloneqq$ i
        }

        \Ciklus{$i \coloneqq 256$-től $1$-ig}{
            j $\coloneqq$ Rand(1, 256)  \\
            temp $\coloneqq$ PermutaciosTabla[$i$] \\
            PermutaciosTabla[$i$] $\coloneqq$ PermutaciosTabla[$j$] \\
            PermutaciosTabla[$j$] $\coloneqq$ temp \\
        }

        \Ciklus{$i \coloneqq 1$-től $256$-ig}{
            PermutaciosTabla[$i + 256$] $\coloneqq$ i
        }
    }
\end{algorithm}

\end{document}